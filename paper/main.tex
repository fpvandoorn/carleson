\documentclass[12pt]{amsart}

\usepackage[utf8]{inputenc}
\usepackage[T1]{fontenc}
\usepackage{lmodern}
\usepackage[margin=1in]{geometry}
\usepackage{amsmath,amssymb,amsthm}
\usepackage{hyperref}
\usepackage{xcolor}
\usepackage{enumitem}

\hypersetup{
  colorlinks = true,
  linkcolor  = blue,
  citecolor  = blue,
  urlcolor   = blue
}


\title{\textbf{Formalizing Carleson's Theorem in Lean}}
\author{Authors}
\date{\today}

\begin{document}

\maketitle

\begin{abstract}
We present the formalization of Carleson's theorem in the proof assistant \emph{Lean}.
This paper describes the mathematical content, organization of the project, the blueprint,
and the main design decisions behind the formalization. It is the result of a large
collaborative effort, written and developed in public.
\end{abstract}

\tableofcontents

\section{Introduction}\label{sec:intro}

\subsection{Brief history of Carleson's theorem and significance}\label{subsec:history}

\subsection{Overview of the formalization project}\label{subsec:overview}

\subsection{Related work}\label{subsec:rel_work}
\begin{itemize}
 	\item Other large-scale formalizations in Lean.
	\item Other harmonic analysis formalizations.
\end{itemize}

\noindent \textit{Acknowledgement.}
The authors acknowledge contributions in the form of small formalization additions,
pointing out corrections to the blueprint,
or supplying ideas to the Lean efforts by the following people:
Michel Alexis,
Bolton Bailey,
Julian Berman,
Joachim Breitner,
Martin Dvoř\'ak,
Georges Gonthier,
Aaron Hill,
Austin Letson,
Bhavik Mehta,
Eric Paul,
Clara Torres,
Dennis Tsar,
Andrew Yang, and
Ruben van de Velde.

L.B., M.I.d.F.F., L.D., F.v.D., M.R., R.S., and C.T. were funded by the Deutsche For\-schungs\-gemein\-schaft (DFG, German Research Foundation) under Germany's Excellence Strategy -- EXC-2047/1 -- 390685813.
L.B. , R.S., and C.T. were also supported by SFB 1060.
A.J. is funded by the T\"UBITAK (Scientific and Technological Research Council of T\"urkiye) under Grant Number 123F122.
J.R. was supported in part by NSF grant DMS-2154835 and a HIM fellowship for the Fall 2024 trimester program in Bonn.

\section{The Metric Space Carleson theorem}\label{sec:main_thm}

\subsection{Mathematical background}\label{subsec:background}
\begin{itemize}
  \item Prerequisites: Fourier series, Real interpolation, Hardy--Littlewood maximal function, doubling measures (\texttt{IsDoubling}), \texttt{WeakType}, \texttt{wnorm}.
\end{itemize}

\subsection{Statement of the main results}\label{subsec:main_results}

\begin{itemize}
	\item Statement of Theorem 1.0.2.
	\item Main propositions in Chapter 2.

\end{itemize}
%\item Level of mathematical detail to include.

\section{Project organization}\label{sec:org}
\begin{itemize}
  \item Project structure and division of tasks.
  \item Floris responsible for lemma statements (pros and cons).
  \item Interaction with harmonic analysis group.
  \item Contributors and roles.
  \item ToMathlib directory and contribution standards.
  \item Communication channels: Zulip, GitHub, Blueprint.
\end{itemize}

\section{Working with a blueprint}\label{sec:blueprint}
\subsection{The blueprint writing process}\label{subsec:blueprint_writing}
\begin{itemize}
	\item Blueprint writing process and finitary arguments.
\end{itemize}

\subsection{Changes and refinements}\label{subsec:blueprint_changes}
\begin{itemize}
	\item $I \leq J$ vs.\ $I \subset J$.
	\item Real interpolation theorem.
	\item Hardy--Littlewood maximal function for finite vs.\ countable balls.
	\item Just one top cube.
	\item Constant tweaking.
\end{itemize}

\subsection{Dealing with mistakes}\label{subsec:blueprint_mistakes}
\begin{itemize}
  \item Lemma 11.1.6.
  \item Lemma 6.3.3/4.
  \item Hölder cancellation (radius $R \to 2R$).
  \item Lemma 6.2.3: additional hypothesis.
\end{itemize}

\subsection{Lessons learned}\label{subsec:blueprint_lessons}
\begin{itemize}
	\item Impact of blueprint choices on errors.
	\item Lack of definition environments $\Rightarrow$ absence from dependency graph.
\end{itemize}

\section{Design decisions}\label{sec:design}
\subsection{Treatment of constants}\label{subsec:constants}

\subsection{The \texttt{ProofData} pattern}\label{subsec:proof_data}

\subsection{Working with real numbers}\label{subsec:real}
\begin{itemize}
	\item \texttt{Real} vs.\ \texttt{NNReal} vs.\ \texttt{ENNReal}:
	\begin{itemize}
		\item Tactic support issues (\texttt{norm\_num}, \texttt{field\_simp}, \texttt{ring}).
	\end{itemize}
\end{itemize}

\subsection{Use of \texttt{ENorm}}\label{subsec:enorm}

\subsection{Working with $L^p$ functions}\label{subsec:Lp}
\begin{itemize}
	\item Working with \texttt{MemLp}, not functions on \texttt{Lp}.
\end{itemize}

\subsection{The \texttt{BoundedCompactSupport} structure}\label{subsec:bdd_comp_supp}
\begin{itemize}
	\item \texttt{BoundedCompactSupport} and packaging conditions.
\end{itemize}

\subsection{The \texttt{fun\_prop} tactic}\label{subsec:fun_prop}
\begin{itemize}
	\item Ongoing experiments with \texttt{fun\_prop}.
\end{itemize}

\subsection{Common pitfalls}\label{subsec:pitfalls}
\begin{itemize}
  \item Using \texttt{Real}.
  \item \texttt{Set.indicator} vs.\ \texttt{Measure.restrict}.
  \item \texttt{Finsets} vs.\ \texttt{Sets} in a \texttt{Fintype}.
\end{itemize}


\section{Conclusion}\label{sec:conclusion}
\begin{itemize}
  \item Project statistics (e.g.\ size of ToMathlib and total project).
  \item Summary of lessons learned:
    \begin{itemize}
      \item Refer to general results early on.
      \item Generalize during blueprint writing.
    \end{itemize}
\end{itemize}

\bibliographystyle{plain}
\bibliography{references}

\end{document}