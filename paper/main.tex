\documentclass[12pt]{amsart}

\usepackage[utf8]{inputenc}
\usepackage[T1]{fontenc}
\usepackage{lmodern}
\usepackage[margin=1in]{geometry}
\usepackage{amsmath,amssymb,amsthm,mathtools}
\usepackage{hyperref}
\usepackage{xcolor}
\usepackage{enumitem}

\usepackage[]{minted}
\newcommand{\lean}[1]{\mintinline{lean}{#1}}

\DeclarePairedDelimiter{\abs}{\lvert}{\rvert}
\newcommand*{\ud}{\, \mathrm{d}}
\newcommand*{\complex}{\mathbb{C}} % set of complex numbers
\newcommand*{\real}{\mathbb{R}} % the set of real numbers
\newcommand{\nnreal}{\real_{\ge0}}
\newcommand{\ennreal}{\overline{\real_{\ge0}}}

% These options can be changed later; these may be personal preferences.
\hypersetup{
  bookmarks,bookmarksopen,bookmarksnumbered,
  colorlinks = true,
  linkcolor  = blue,
  citecolor  = red,
  urlcolor   = grey,
  breaklinks
}
% maxbibnames: List all authors in the bibliography for up to 5 authors. Use up to 4 names for determining the label. Choose alphabetical style.
\usepackage[backend=biber,maxbibnames=5,maxalphanames=4,style=alphabetic]{biblatex}
\usepackage[autostyle]{csquotes}
\addbibresource{references.bib}

\theoremstyle{plain}
\newtheorem{theorem}{Theorem}[subsection]

\newcommand{\N}{{\mathbb{N}}}
\newcommand{\Z}{{\mathbb{Z}}}
\newcommand{\Q}{{\mathbb{Q}}}
\newcommand{\R}{{\mathbb{R}}}

\DeclareMathOperator{\ch}{\operatorname{ch}}
\DeclareMathOperator{\dens}{\operatorname{dens}}
\DeclareMathOperator{\supp}{\operatorname{supp}}
\DeclareMathOperator{\tp}{\operatorname{top}}
\DeclareMathOperator{\im}{\operatorname{im}}
\DeclareMathOperator{\Lip}{\operatorname{Lip}}
\DeclareMathOperator{\bd}{\operatorname{bd}}
\DeclareMathOperator*{\esssup}{\operatorname{ess\,sup}}

\def\C{\mathbb{C}}
\newcommand{\pc}{{\mathrm{c}}}
\newcommand{\ps}{{\mathrm{s}}}
\newcommand{\AD}{{\bf s}}
\newcommand{\fc}{{\Omega}}
\newcommand{\borel}{{\mathcal{E}}}
\newcommand{\borelb}{{\mathcal{G}}}
\newcommand{\scI}{{\mathcal{I}}} % note: renamed because it conflicted with another command
\newcommand{\tQ}{{Q}}
\newcommand{\mfa}{{\vartheta}}
\newcommand{\mfb}{{\theta}}
\newcommand{\Mf}{{\Theta}}
\newcommand{\fcc}{{\mathcal{Q}}}

\title{\textbf{Formalizing Carleson's Theorem in Lean}}
\author{Authors}
\date{\today}

\begin{document}

\maketitle

\begin{abstract}
We present the formalization of Carleson's theorem in the proof assistant \emph{Lean}.
This paper describes the mathematical content, organization of the project, the blueprint,
and the main design decisions behind the formalization. It is the result of a large
collaborative effort, written and developed in public.
\end{abstract}

\tableofcontents

\section{Introduction}\label{sec:intro}

\subsection{Brief history of Carleson's theorem and significance}\label{subsec:history}

\subsection{Overview of the formalization project}\label{subsec:overview}

\subsection{Related work}\label{subsec:rel_work}
\begin{itemize}
 	\item Other large-scale formalizations in Lean.
	\item Other harmonic analysis formalizations.
\end{itemize}

\noindent \textit{Acknowledgement.}
The authors acknowledge contributions in the form of small formalization additions,
pointing out corrections to the blueprint,
or supplying ideas to the Lean efforts by the following people:
Michel Alexis,
Bolton Bailey,
Julian Berman,
Joachim Breitner,
Martin Dvoř\'ak,
Georges Gonthier,
Aaron Hill,
Austin Letson,
Bhavik Mehta,
Eric Paul,
Clara Torres,
Dennis Tsar,
Andrew Yang, and
Ruben van de Velde.

L.B., M.I.d.F.F., L.D., F.v.D., M.R., R.S., and C.T. were funded by the Deutsche For\-schungs\-gemein\-schaft (DFG, German Research Foundation) under Germany's Excellence Strategy -- EXC-2047/1 -- 390685813.
L.B., R.S., and C.T. were also supported by SFB 1060.
A.J. is funded by the T\"UBITAK (Scientific and Technological Research Council of T\"urkiye) under Grant Number 123F122.
J.R. was supported in part by NSF grant DMS-2154835 and a HIM fellowship for the Fall 2024 trimester program in Bonn.

\section{The Metric Space Carleson theorem}\label{sec:main_thm}

\subsection{Mathematical background}\label{subsec:background}
\begin{itemize}
  \item Prerequisites: Fourier series, Real interpolation, Hardy--Littlewood maximal function, doubling measures (\texttt{IsDoubling}), \texttt{WeakType}, \texttt{wnorm}.
\end{itemize}

\subsection{Statement of the main results}\label{subsec:main_results}

\subsubsection{Classical Carleson}\label{subsubsec:classical-result}

The namesake of the formalization project is the classical result of Carleson on the pointwise convergence of Fourier series.
The specific statement is as follows:

\begin{theorem}[classical Carleson]
    \label{classical-carleson}
    Let $f$ be a $2\pi$-periodic complex-valued continuous function on $\mathbb{R}$.
    Then for almost all $x \in \mathbb{R}$ we have
    \begin{equation}\label{eq:fourier-limit}
      \lim_{N\to\infty}S_N f(x) = f(x),
    \end{equation}
    where $S_N f$ is the $N$-th partial Fourier sum $S_N(x):= \sum_{n=-N}^N c_n e^{inx}$
    with coefficients $c_n:=\frac 1{2\pi}\int_0^{2\pi}f(x) e^{- i nx}\, dx.$
\end{theorem}

The formalization was approached through a recent result (blabla reference) which encompasses a number of generalizations of Carleson's original theorem.
The context for this new result is called a \emph{doubling metric measure space}, accompanied by a \emph{cancellative compatible collection} of real valued continuous functions.
For a \emph{one-sided} respectively \emph{two-sided Calder\'on--Zygmund kernel} $K$ on this space, one obtains bounds for certain \emph{generalized Carleson operators} involving the collection of functions and the kernel.
Specific choices of these objects and structures are leveraged to derive the classical Carleson theorem.


\subsubsection{Preliminary definitions}\label{subsubsec:definitions}

A doubling metric measure space $(X,\rho,\mu, a)$ is a complete and locally compact metric space $(X,\rho)$
equipped with a non-zero locally finite Borel measure $\mu$ that satisfies the doubling condition that for all $x\in X$ and all $R>0$ we have
\begin{equation}\label{doublingx}
    \mu(B(x,2R))\le 2^a\mu(B(x,R))\,,
\end{equation}
where we have denoted by $B(x,R)$ the open ball of radius $R$ centered at $x$.
%\begin{equation}\label{eq-define-ball}
%    B(x,R):=\{y\in X: \rho(x,y)<R\}.
%\end{equation}
%In a doubling metric measure space the closed balls are compact and $\mu$ is positive on all non-empty open sets.

A collection $\Mf$ of real valued continuous functions on the doubling metric measure space $(X,\rho,\mu,a)$ is called compatible,
if there is a point $o\in X$ where all the functions are equal to $0$,
and if there exists for each ball $B \subset X$ a metric $d_B$ on $\Mf$,
such that the following five properties \eqref{osccontrol}, \eqref{firstdb}, \eqref{monotonedb}, \eqref{seconddb}, and \eqref{thirddb} are satisfied.
For every ball $B \subset X$
\begin{equation}\label{osccontrol}
    \sup_{x,y\in B}|\mfa(x)-{\mfa(y)}- \mfb(x)+{\mfb(y)}| \le d_{B}(\mfa,\mfb)\,.
\end{equation}
For any two balls $B_1=B(x_1,R)$, $B_2= B(x_2,2R)$ in $X$ with $x_1\in B_2$ and any $\mfa,\mfb\in \Mf$,
\begin{equation}\label{firstdb}
    d_{B_2}(\mfa,\mfb) \le 2^a d_{B_1}(\mfa,\mfb) .
\end{equation}
For any two balls $B_1, B_2$ in $X$ with $B_1 \subset B_2$ and any $\mfa, \mfb \in \Mf$
\begin{equation}\label{monotonedb}
    d_{B_1}(\mfa,\mfb) \le d_{B_2}(\mfa, \mfb)
\end{equation}
and for any two balls
$B_1=B(x_1,R)$, $B_2= B(x_2,2^aR)$
with $B_1\subset B_2$, and $\mfa,\mfb\in \Mf$,
\begin{equation}\label{seconddb}
    2d_{B_1}(\mfa,\mfb) \le d_{B_2}(\mfa,\mfb) .
\end{equation}
For every ball $B$ in $X$ and every $d_B$-ball $\tilde B$ of radius $2R$ in $\Mf$, there is a collection $\mathcal{B}$ of
at most $2^a$ many $d_B$-balls of radius $R$ covering $\tilde B$, that is,
\begin{equation}\label{thirddb}
    \tilde B\subset \bigcup \mathcal{B}.
\end{equation}

Further, a compatible collection $\Mf$ is called cancellative, if
for any ball $B$ in $X$ of radius $R$, any Lipschitz function $\varphi: X\to \C$
supported on $B$, and any $\mfa,\mfb\in \Mf$ we have
\begin{equation}
    \label{eq-vdc-cond}
    |\int_B e(\mfa(x)-{\mfb(x)}) \varphi(x) d\mu(x)|\le 2^a \mu(B)\|\varphi\|_{\Lip(B)}
(1+d_B(\mfa,\mfb))^{-\frac{1}{a}},
\end{equation}
where $\|\cdot\|_{\Lip(B)}$ denotes the inhomogeneous Lipschitz norm on $B$:
$$
    \|\varphi\|_{\Lip(B)} = \sup_{x \in B} |\varphi(x)| + R \sup_{x,y \in B, x \neq y} \frac{|\varphi(x) - \varphi(y)|}{\rho(x,y)}\,.
$$

A one-sided Calder\'on--Zygmund kernel $K$ on the doubling metric measure space $(X, \rho, \mu, a)$
is a measurable function $K:X\times X\to \mathbb{C}$ such that for all $x,y',y\in X$ with $x\neq y$, we have
\begin{equation}\label{eqkernel-size}
    |K(x,y)| \leq \frac{2^{a^3}}{V(x,y)}
\end{equation}
and if $2\rho(y,y') \leq \rho(x,y)$, then
\begin{equation}
  \label{eqkernel-y-smooth}
  |K(x,y) - K(x,y')| \leq \left(\frac{\rho(y,y')}{\rho(x,y)}\right)^{\frac{1}{a}}\frac{2^{a^3}}{V(x,y)},
\end{equation}
where $V(x,y):=\mu(B(x,\rho(x,y)))$.

A two-sided Calder\'on--Zygmund kernel is a one-sided Calder\'on--Zygmund kernel $K$ which additionally satisfies for all $x,x',y\in X$ with $x\neq y$ and $2\rho(x,x') \leq \rho(x,y)$,
\begin{equation}
    \label{eqkernel-x-smooth}
    |K(x,y) - K(x',y)| \leq \left(\frac{\rho(x,x')}{\rho(x,y)}\right)^{\frac{1}{a}}\frac{2^{a^3}}{V(x,y)}.
\end{equation}

Let $f: X \to \C$ be a bounded, measurable function supported on a set of finite measure.
Define the maximally truncated non-tangential singular integral $T_{*}$ associated with $K$ by
\begin{equation}
    \label{def-tang-unm-op}
    T_{*}f(x):=\sup_{R_1 < R_2} \sup_{\rho(x,x')<R_1} \left|\int_{R_1< \rho(x',y) < R_2} K(x',y) f(y) \, \mathrm{d}\mu(y) \right|\,.
\end{equation}
Additionally, define for $r > 0$, $x\in X$:
\begin{equation}
\label{def-T-r}
T_r f(x):= \int_{r\le\rho(x,y)} K(x,y) f(y) \, d\mu(y) = \int_{X\setminus B(x,r)} K(x,y) f(y) \, d\mu(y).
\end{equation}
Finally, define the generalized Carleson operator $T$ by
\begin{equation}
    \label{def-main-op}
    Tf(x):=\sup_{\mfa\in\Mf} \sup_{0 < R_1 < R_2}\left| \int_{R_1 < \rho(x,y) < R_2} K(x,y) f(y) e(\mfa(y)) \, \mathrm{d}\mu(y) \right|\, ,
\end{equation}
where $e(r)=e^{ir}$.

For a Borel function $\tQ:X\to \Mf$, and $\mfa \in \Mf$ and $x\in X$ define
\begin{equation}
    R_{\tQ}(\mfa,x)=\sup\{r>0:d_{B(x,r)}(\mfa, \tQ(x))<1\}
\end{equation}
and define further
\begin{equation}
    \label{def-lin-star-op}
    T_{\tQ}^\mfa f(x):=\sup_{R_1<R_2} \ \sup_{\rho(x,x')<R_1}
    \left|\int_{R_1< \rho(x',y) < \min\{R_2, R_{\tQ}(\mfa,x')\}} K(x',y) f(y) \, \mathrm{d}\mu(y) \right|
\end{equation}
Define further the linearized generalized Carleson operator $T_\tQ$ by
\begin{equation}
    \label{def-lin-main-op}
    T_\tQ f(x):= \sup_{0 < R_1 < R_2}\left| \int_{R_1 < \rho(x,y) < R_2} K(x,y) f(y) e(\tQ(x)(y)) \, \mathrm{d}\mu(y) \right|\, ,
\end{equation}
where again $e(r)=e^{ir}$.


\subsubsection{Metric space Carleson}\label{subsubsec:metric-carleson}

Our first main result is the following restricted weak type estimate for $T$ in the range $1<q\le 2$,
which by interpolation techniques recovers $L^q$ estimates for the open range $1<q<2$.
\begin{theorem}[metric space Carleson]
\label{metric-space-Carleson}
    For all integers $a \ge 4$ and real numbers $1<q\le 2$ the following holds.
    Let $(X,\rho,\mu,a)$ be a doubling metric measure space.
    Let $\Mf$ be a cancellative compatible collection of functions and let $K$ be a one-sided Calder\'on--Zygmund kernel on $(X,\rho,\mu,a)$.
    Assume that for every bounded measurable function $g$ on $X$ supported on a set of finite measure we have
    \begin{equation}\label{nontanbound}
        \|T_{*}g\|_{2} \leq 2^{a^3} \|g\|_2\,,
    \end{equation}
    where $T_{*}$ is defined in \eqref{def-tang-unm-op}.
    Then for all Borel sets $F$ and $G$ in $X$ and all Borel functions $f:X\to \C$ with
    $|f|\le \mathbf{1}_F$, we have, with $T$ defined in \eqref{def-main-op},
    \begin{equation}
        \label{resweak}
        \left|\int_{G} T f \, \mathrm{d}\mu\right| \leq \frac{2^{443a^3}}{(q-1)^6} \mu(G)^{1-\frac{1}{q}} \mu(F)^{\frac{1}{q}}\, .
    \end{equation}
\end{theorem}

In some applications, such as the Walsh-case of Carleson's theorem, the kernel $K$ naturally depends also on the modulation functions $\mfa$.
The fact that we don't assume H\"older-continuity of the kernel $K$ in the first argument allows us to also capture this situation.

We now state another Theorem with a slightly weaker replacement of the assumption \eqref{nontanbound}.
It implies the Walsh-case of Carleson's theorem.

\begin{theorem}[linearised metric Carleson]
\label{linearised-metric-Carleson}
    For all integers $a \ge 4$ and real numbers $1<q\le 2$ the following holds.
    Let $(X,\rho,\mu,a)$ be a doubling metric measure space. Let $\Mf$ be a
    cancellative compatible collection of functions.
    Let $\tQ:X\to \Mf$ be a Borel function with finite range.
    Let $K$ be a one-sided Calder\'on--Zygmund kernel on $(X,\rho,\mu,a)$. Assume that for every $\mfa\in \Mf$ and every bounded measurable function $g$ on $X$ supported on a set of finite measure we have
    \begin{equation}\label{linnontanbound}
        \|T_{\tQ}^\mfa g\|_{2} \leq 2^{a^3} \|g\|_2\,,
    \end{equation}
    where $T_{\tQ}^\mfa$ is defined in \eqref{def-lin-star-op}.
    Then for all Borel sets $F$ and $G$ in $X$ and all Borel functions $f:X\to \C$ with
    $|f|\le \mathbf{1}_F$, we have, with $T_\tQ$ defined in \eqref{def-lin-main-op},
    \begin{equation}
        \label{linresweak}
        \left|\int_{G} T_\tQ f \, \mathrm{d}\mu\right| \le \frac{2^{443a^3}}{(q-1)^6} \mu(G)^{1-\frac{1}{q}} \mu(F)^{\frac{1}{q}}\, .
    \end{equation}
\end{theorem}

For two-sided Calder\'on--Zygmund kernels, the additional regularity allows us to weaken one of the assumptions to the family of operators $T_r$, which is easier to work with in applications.
In particular the two-sided case can be applied to derive the classical Carleson's theorem.

\begin{theorem}[two-sided metric space Carleson]
    \label{two-sided-metric-space-Carleson}
        For all  integers $a \ge  4$ and real numbers $1<q\le 2$ the following holds.
        Let $(X,\rho,\mu,a)$ be a doubling metric measure space. Let  $\Mf$ be a
        cancellative compatible  collection of functions and let $K$ be a two-sided Calder\'on--Zygmund kernel on $(X,\rho,\mu,a)$. Assume  that for every bounded measurable function $g$ on $X$ supported on a set of finite measure and all $r>0$ we have
      \begin{equation}\label{two-sided-Hr-bound-assumption}
            \|T_r g\|_{2} \leq 2^{a^3} \|g\|_2\,.
        \end{equation}
        Then for all Borel sets $F$ and $G$ in $X$ and
        all Borel functions $f:X\to \C$ with
        $|f|\le \mathbf{1}_F$, we have, with $T$ defined in \eqref{def-main-op},
      \begin{equation}
        \label{two-sided-resweak}
            \left|\int_{G} T f \, \mathrm{d}\mu\right| \leq \frac{2^{474a^3}}{(q-1)^6} \mu(G)^{1-\frac{1}{q}} \mu(F)^{\frac{1}{q}}\, .
        \end{equation}
\end{theorem}

\subsubsection{Metric space Carleson applied to the real line}\label{subsubsec:metric-carleson-real}

In order to arrive at the classical result \Cref{classical-carleson}, we use the following specific instance of the main result.

Let $\kappa:\R\to \C$ be the function defined by
$\kappa(0)=0$ and for $0<|x|<1$
\begin{equation}\label{eq-hilker}
\kappa(x)=\frac { 1-|x|}{1-e^{ix}}\,
\end{equation}
and for $|x|\ge 1$,
\begin{equation}\label{eq-hilker1}
\kappa(x)=0\, .
\end{equation}
Note that this function is continuous at every point $x$ with $|x|>0$.

\begin{theorem}[real Carleson]\label{real-Carleson}
    Let $F,G$ be Borel subsets of $\R$ with finite measure. Let $f$ be a bounded measurable function on $\R$ with $|f|\le \mathbf{1}_F$. Then
\begin{equation}
    \left|\int _G Tf(x) \, dx\right| \le C_{4,2} |F|^{\frac 12} |G|^{\frac 12} \, ,
\end{equation}
where
\begin{equation}
    \label{define-T-carleson}
    T f(x)=\sup_{n\in \mathbb{Z}}
    \sup_{r>0}\left|\int_{r<|x-y|<1} f(y)\kappa(x-y) e^{iny}\, dy\right|\, .
\end{equation}
\end{theorem}

%\begin{itemize}
%	\item Statement of Theorem 1.0.2.
%	\item Main propositions in Chapter 2.
%
%\end{itemize}
%\item Level of mathematical detail to include.

\section{Project organization}\label{sec:org}
\begin{itemize}
  \item Project structure and division of tasks.
  \item Floris responsible for lemma statements (pros and cons).
  \item Interaction with harmonic analysis group.
  \item Contributors and roles.
  \item ToMathlib directory and contribution standards.
  \item Communication channels: Zulip, GitHub, Blueprint.
\end{itemize}

\section{Working with a blueprint}\label{sec:blueprint}
\subsection{The blueprint writing process}\label{subsec:blueprint_writing}
\begin{itemize}
	\item Blueprint writing process and finitary arguments.
\end{itemize}

\subsection{Changes and refinements}\label{subsec:blueprint_changes}

Throughout the formalisation process, most of the blueprint only needed minor changes, such as fixing typos of clarifications of minor details. Let us highlight four aspects where formalisation resulted in more substantial changes or refinements.

\subsubsection{$I \leq J$ vs.\ $I \subset J$}
To be written!

\subsubsection{Just one top cube} To be written!

\subsubsection{The Hardy--Littlewood maximal function} One main ingredient in the proof is a Vitali covering argument, using the Hardy--Littlewood maximal function.
To recall: given $f\colon\real^d\to\complex$, the Hardy–Littlewood maximal function $M f$ of $f$ is usually defined as
\begin{equation*}
Mf\colon\real \to\ennreal, \quad x\mapsto \sup_{r>0}\frac{1}{|B(x,r)|}\int_{B(x,r)}|f(y)|\,dy.
\end{equation*}
If $f$ is integrable, then $M f$ is (weak) $L^1$, and hence finite almost everywhere. Initially, the blueprint only used a variant taking a supremum over finitely many balls: this ensured the resulting function was \emph{always} finite, hence real-valued. Applying it to a countable collection of balls required taking a limit and using the monotone convergence theorem.
This is a somewhat inelegant solution: most results about the maximal function do hold for any suprema, hence should be proven in the appropriate generality. We also prefer to avoid a superfluous limiting argument.

For the formal proof, we changed the definition to allow arbitrary suprema: this means the maximal function is $[0,\infty]$-valued. Being able to speak of $L^p$-functions valued in $[0,\infty]$ prompted the introduction of \emph{extended norms}, see Subsection~\ref{subsec:enorm}.

\subsubsection{The real interpolation theorem}
The Marcinkiewicz real interpolation theorem is a standard result in functional analysis.
It is also a key prerequisite for the proof of Carleson's theorem: for example, it is applied to the Hardy–Littlewood maximal function to deduce it is of strong type $L^p$, for $p\in(0,\infty)$.
Our formalised version aims to reduce superfluous assumptions commonly found in the mathematical literature. For instance, it applies to any sub-additive function (not merely sub-linear ones) --- that is, there is no need for a scalar multiplication on the target. Similarly, any positive exponent $p$ is allowed (as opposed to demanding $p\geq 1$).

In light of generalising the Hardy--Littlewood maximal function, this proof had to be generalised again:
applying it to the maximal functions requires a version which applies to functions with co-domain $[0, \infty]$ (or, more generally, any commutative monoid with a compatible enorm).
In return for a significant amount of refactoring work\footnote{the initial formalisation was about 5000 lines long; the generalisation touched about 2000 of them},
we have formalised a more general argument that is also conceptually clearer: for instance, it demonstrates that the subtractive structure on the co-domain is not used in a meaningful way.
For further detail, we refer the reader to a separate article \cite{PortegiesRothgangRealInterpolation}.

\subsection{Constant tweaking} To be written!

\subsection{Dealing with mistakes}\label{subsec:blueprint_mistakes}
\begin{itemize}
  \item Lemma 11.1.6.
  \item Lemma 6.3.3/4.
  \item Hölder cancellation (radius $R \to 2R$).
  \item Lemma 6.2.3: additional hypothesis.
\end{itemize}

\subsection{Lessons learned}\label{subsec:blueprint_lessons}
\begin{itemize}
	\item Impact of blueprint choices on errors.
	\item Lack of definition environments $\Rightarrow$ absence from dependency graph.
\end{itemize}

\section{Design decisions}\label{sec:design}
\subsection{Treatment of constants}\label{subsec:constants}

\subsection{The \texttt{ProofData} pattern}\label{subsec:proof_data}

\subsection{Working with real numbers}\label{subsec:real}
\begin{itemize}
	\item \texttt{Real} vs.\ \texttt{NNReal} vs.\ \texttt{ENNReal}:
	\begin{itemize}
		\item Tactic support issues (\texttt{norm\_num}, \texttt{field\_simp}, \texttt{ring}).
	\end{itemize}
\end{itemize}

\subsection{Use of \texttt{ENorm}}\label{subsec:enorm}

This project motivated the creation of a new formalisation abstraction: \emph{extended norms} (\lean{enorm}s for short) generalise many results from normed spaces to $\ennreal$.
Their introduction was motivated by defining the Hardy--Littlewood maximal with co-domain $\ennreal=[0,\infty]$ (see Subsection~\ref{subsec:blueprint_changes}).
Previously, a statement ``$f\in L^p$'' was only defined for functions $f\colon X\to E$ from a measure space $X$ to a normed space $E$. Speaking about the maximal function being in weak $L^p$ requires a notion of $L^p$ functions with target $\ennreal$. Fortunately, many basic facts about $L^p$ functions do not require the target to be a normed space and have reasonable analogues for $\ennreal$: for instance, the $L^p$ norm of $f\colon X\to\ennreal$ for $p<\infty$ can be defined as $(\int_X \abs{f(x)}^p \ud x)^{1/p}$, where the integrand is the function $X\to[0,\infty], x\mapsto \abs{f(x)}^p$.

Motivated by this observation, we introduced a new notation class \lean{ENorm}, describing a type endowed with a function $\texttt{enorm}\colon X\to\ennreal$.
This captures both normed spaces and $\ennreal$. The enorm of a normed space is the ambient norm, considered as a function into $\ennreal$; the enorm on $\ennreal$ is the identity function.
Some definitions only require the existence of an enorm; many results require suitable compatibility conditions. %(For instance, on an abelian monoid $\alpha$ endowed with an enorm, we may require the enorm to respect the triangle inequality with respect to addition and be positive definite, or ask for the enorm to be continuous.)
Most lemmas necessary in this project only require an \lean{ENormedAddCommMonoid}, i.e.\ an abelian monoid endowed with a continuous enorm that is positive definite and satisfies the triangle inequality. Many lemmas, in fact, only require slightly weaker assumptions. % for instance, don't require positive definiteness

Pursuing this approach required generalising all necessary definitions to \lean{enorm}s, as well as all basic results that are required in this project. This was a significant effort, but the verified nature of formalisation helped a lot by allowing to do so incrementally: in a first step, the new definition was added (together with the enorm instance on normed spaces). Then, all basic definitions were changed to allow an \lean{enorm} as codomain, whenever this did not require further changes. A third step modified many lemma statements to use \lean{enorm}s instead of norms.
%; while massive, this actually led to simplifications.
The final phase involved generalising lemmas: often, this required small changes (such as, constants changing from a real to an extended real number, or adding a hypothesis about some quantity in $\ennreal$ being finite). At each point, verification ensured correctness: if all proofs still compiled, no more changes needed to be made.

To illustrate the magnitude of this refactoring, let us give some statistics: phase two (changing the basic definitions) changed about 300 lines of code; phase three (changing lemma statements to enorms) about 1000, and the lemma generalisations so far touched about 2500 lines of code.

The last phase (generalising mathlib lemmas) is not exhaustive: while substantial parts of mathlib have been generalised, some areas would require further refactor (and will follow as needed). To give an example, the total variation of a path in $\ennreal$ is a sensible quantity, even though $\ennreal$ is not a metric space. Introducing a weaker notion of extended metric spaces will enable new applications, and also provide the proper abstraction to state many scalar multiplication lemmas nicely. This is an ongoing effort by Felix Pernegger.

The process of generalisation revealed that this abstraction is also useful on its own: previously, many proofs in mathlib involved converting between norms taking values in $\real$ or $\nnreal$; using enorms provides a useful alternative. At the same time, it illustrates how the right abstractions can avoid duplicating work, while providing a useful clarification to mathematics.

\subsection{Working with $L^p$ functions}\label{subsec:Lp}
\begin{itemize}
	\item Working with \texttt{MemLp}, not functions on \texttt{Lp}.
\end{itemize}

\subsection{The \texttt{BoundedCompactSupport} structure}\label{subsec:bdd_comp_supp}
\begin{itemize}
	\item \texttt{BoundedCompactSupport} and packaging conditions.
	\item Ongoing experiments with \texttt{fun\_prop}. TODO: consider whether this should be a separate subsection.
\end{itemize}

\subsection{Common pitfalls}\label{subsec:pitfalls}
\begin{itemize}
  \item Using \texttt{Real}.
  \item \texttt{Set.indicator} vs.\ \texttt{Measure.restrict}.
  \item \texttt{Finsets} vs.\ \texttt{Sets} in a \texttt{Fintype}.
\end{itemize}


\section{Conclusion}\label{sec:conclusion}
\begin{itemize}
  \item Project statistics (e.g.\ size of ToMathlib and total project).
  \item Summary of lessons learned:
    \begin{itemize}
      \item Refer to general results early on.
      \item Generalize during blueprint writing.
    \end{itemize}
\end{itemize}

%\bibliographystyle{plain}
%\bibliography{references}
\printbibliography

\end{document}