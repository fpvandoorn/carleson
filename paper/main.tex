\documentclass[12pt]{amsart}

\usepackage[utf8]{inputenc}
\usepackage[T1]{fontenc}
\usepackage{lmodern}
\usepackage[margin=1in]{geometry}
\usepackage{amsmath,amssymb,amsthm}
\usepackage{hyperref}
\usepackage{xcolor}
\usepackage{enumitem}

\hypersetup{
  colorlinks = true,
  linkcolor  = blue,
  citecolor  = blue,
  urlcolor   = blue
}


\title{\textbf{Formalizing Carleson's Theorem in Lean}}
\author{Authors}
\date{\today}

\begin{document}

\maketitle

\begin{abstract}
We present the formalization of Carleson's theorem in the proof assistant \emph{Lean}.
This paper describes the mathematical content, organization of the project, the blueprint,
and the main design decisions behind the formalization. It is the result of a large
collaborative effort, written and developed in public.
\end{abstract}

\tableofcontents

\section{Introduction}
\begin{itemize}
  \item Brief history of Carleson's theorem and significance.
  \item Overview of the formalization project.
  \item Related work:
    \begin{itemize}
      \item Other large-scale formalizations in Lean.
      \item Other harmonic analysis formalizations.
    \end{itemize}
\end{itemize}

\section{Mathematics}
\begin{itemize}
  \item Prerequisites: Fourier series, Real interpolation, Hardy--Littlewood maximal function, doubling measures (\texttt{IsDoubling}), \texttt{WeakType}, \texttt{wnorm}.
  \item Statement of Theorem 1.0.2.
  \item Main propositions in Chapter 2.
  \item Level of mathematical detail to include.
\end{itemize}

\section{Organization}
\begin{itemize}
  \item Project structure and division of tasks.
  \item Floris responsible for lemma statements (pros and cons).
  \item Interaction with harmonic analysis group.
  \item Contributors and roles.
  \item ToMathlib directory and contribution standards.
  \item Communication channels: Zulip, GitHub, Blueprint.
\end{itemize}

\section{Blueprint}
\begin{itemize}
  \item Blueprint writing process and finitary arguments.
  \item Changes and refinements:
    \begin{itemize}
      \item $I \leq J$ vs.\ $I \subset J$.
      \item Real interpolation theorem.
      \item Hardy--Littlewood maximal function for finite vs.\ countable balls.
      \item Just one top cube.
      \item Constant tweaking.
    \end{itemize}
  \item Mistakes and lessons learned:
    \begin{itemize}
      \item Lemma 11.1.6.
      \item Lemma 6.3.3/4.
      \item Hölder cancellation (radius $R \to 2R$).
      \item Lemma 6.2.3: additional hypothesis.
    \end{itemize}
  \item Impact of blueprint choices on errors.
  \item Lack of definition environments $\Rightarrow$ absence from dependency graph.
\end{itemize}

\section{Design Decisions}
\begin{itemize}
  \item Use of \texttt{ENorm}.
  \item Explicit constants \texttt{C\_a\_b\_c}.
  \item \texttt{ProofData} pattern.
  \item Working with \texttt{MemLp}, not functions on \texttt{Lp}.
  \item \texttt{Real} vs.\ \texttt{NNReal} vs.\ \texttt{ENNReal}:
    \begin{itemize}
      \item Tactic support issues (\texttt{norm\_num}, \texttt{field\_simp}, \texttt{ring}).
    \end{itemize}
  \item \texttt{BoundedCompactSupport} and packaging conditions.
  \item Ongoing experiments with \texttt{fun\_prop}.
  \item Common pitfalls:
    \begin{itemize}
      \item Using \texttt{Real}.
      \item \texttt{Set.indicator} vs.\ \texttt{Measure.restrict}.
      \item \texttt{Finsets} vs.\ \texttt{Sets} in a \texttt{Fintype}.
    \end{itemize}
\end{itemize}

\section{Conclusion}
\begin{itemize}
  \item Project statistics (e.g.\ size of ToMathlib and total project).
  \item Summary of lessons learned:
    \begin{itemize}
      \item Refer to general results early on.
      \item Generalize during blueprint writing.
    \end{itemize}
\end{itemize}

\bibliographystyle{plain}
\bibliography{references}

\end{document}