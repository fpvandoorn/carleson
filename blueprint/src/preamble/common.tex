% !TeX root = print.tex
% Put any macro and import needed for the project here.
% This will be used by both the web and print versions of the blueprint.
% This file is not meant to be built. Build src/web.tex or src/print.text instead.

\usepackage{mathtools}
\usepackage{amssymb}
\usepackage{amsthm}
\usepackage{amsmath}
\usepackage{graphicx}
\usepackage{xcolor}
\usepackage{enumerate}
\usepackage[hidelinks, colorlinks=false]{hyperref}
% \usepackage{showlabels} % not compatible with Plastex
\usepackage[normalem]{ulem}

\newcommand{\rs}[1]{{\color{blue}  RS: #1.}}
\newcommand{\lars}[1]{{\color{red}  LB: #1.}}
\newcommand{\asgar}[1]{{\color{green} AJ: #1}} %\color{TealBlue}
\newcommand{\ct}[1]{{\color{purple}  CT: #1}}


\theoremstyle{plain}
\newtheorem{theorem}{Theorem}[section]
\newtheorem{lemma}[theorem]{Lemma}
\newtheorem{proposition}[theorem]{Proposition}
\newtheorem{cor}[theorem]{Corollary}
\theoremstyle{definition}
\newtheorem{definition}[theorem]{Definition}
\newtheorem{remark}[theorem]{Remark}
\newtheorem{example}[theorem]{Example}
\newtheorem{examples}[theorem]{Examples}
\numberwithin{section}{chapter}
\numberwithin{subsection}{section}
\numberwithin{subsubsection}{subsection}
\numberwithin{equation}{section}

\newcommand{\N}{{\mathbb{N}}}
\newcommand{\Z}{{\mathbb{Z}}}
\newcommand{\Q}{{\mathbb{Q}}}
\newcommand{\R}{{\mathbb{R}}}
% In some distributions, \C is already defined so \newcommand fails
\def\C{\mathbb{C}}
\DeclareMathOperator{\ch}{\operatorname{ch}}
\DeclareMathOperator{\dens}{\operatorname{dens}}
\DeclareMathOperator{\supp}{\operatorname{supp}}
\DeclareMathOperator{\tp}{\operatorname{top}}
\DeclareMathOperator{\im}{\operatorname{im}}
\DeclareMathOperator{\Lip}{\operatorname{Lip}}
\DeclareMathOperator{\bd}{\operatorname{bd}}
\DeclareMathOperator*{\esssup}{\operatorname{ess\,sup}}

\newcommand{\fp}{{\mathfrak p}}
\newcommand{\fP}{{\mathfrak P}}
\newcommand{\fu}{{\mathfrak u}}
\newcommand{\fU}{{\mathfrak U}}
\newcommand{\fv}{{\mathfrak v}}
\newcommand{\fq}{{\mathfrak q}}
\newcommand{\fQ}{{\mathfrak Q}}
\newcommand{\fT}{{\mathfrak T}}
\newcommand{\fL}{{\mathfrak L}}
\newcommand{\fC}{{\mathfrak C}}
\newcommand{\pc}{{\mathrm{c}}}
\newcommand{\ps}{{\mathrm{s}}}
\newcommand{\AD}{{\bf s}}
\newcommand{\fc}{{\Omega}}
\newcommand{\borel}{{\mathcal{E}}}
\newcommand{\borelb}{{\mathcal{G}}}
\newcommand{\scI}{{\mathcal{I}}} % note: renamed because it conflicted with another command
\newcommand{\tQ}{{Q}}
\newcommand{\mfa}{{\vartheta}}
\newcommand{\mfb}{{\theta}}
\newcommand{\Mf}{{\Theta}}
\newcommand{\fcc}{{\mathcal{Q}}}
